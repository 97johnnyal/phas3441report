\section{Simulation (JAL, CKG, GA, AS)}
The simulation team aimed to construct an interactive software package that could simulate and visualise the motion of droplets bouncing on a liquid surface. The goal here was to cover aspects of quantum mechanics that would not be displayed by the prototype, such as double slit diffraction and tunnelling effects.
\subsection{Initial work in Python}
Initially, work undertaken by the simulation group consisted of finding suitable mechanical descriptions of the droplet's motion, and implementing those in code. A simplified model was taken from \cite{brady2014bouncing}, where the droplet was treated as stationary in the x-y plane and moving in z according to (\ref{equ:basicHeight}). Here, $h_0$ is the maximum height of the drop, $\omega_0$ is the driving frequency of the system, r is the displacement of the drop from the centre in polar co-ordinates and c is the speed of the wave. $J_0$ is a Bessel function of the first kind. With $\omega_0$, r and c set arbitrarily to 1, wave motion was demonstrated using an animation framework from \cite{waveanimation}. The results of this are shown in Figure \ref{fig:basicAnimation}. Unfortunately, the Python language used to generate this proved too slow to be usable as a live demonstration, so any simulations generated using this method would need to be exported to a movie file and played back later. Initially, this was deemed an unacceptable solution, and so Java was chosen as a more efficient language to use in future.

\begin{equation}
    h = -h_0 \cos{(\omega_0 t)} J_0 (\omega_0 r/c)
    \label{equ:basicHeight}
\end{equation}

\begin{figure}[h]
    \begin{subfigure}{0.5\textwidth}
        \centering
        \includegraphics[width=\linewidth]{simulation/basich0.png}
        \caption{Wave motion at $h=0$}
    \end{subfigure}
    \begin{subfigure}{0.5\textwidth}
        \centering
        \includegraphics[width=\linewidth]{simulation/basichmax.png}
        \caption{Wave motion at $h=h_0$}
    \end{subfigure}
\caption{Simulation output generated with Python and Matplotlib. (a) represents the initial state of the wavefield, while (b) represents the state of the wave-field when the "droplet" at the centre is at its maximum height}
\label{fig:basicAnimation}
\end{figure}


\subsection{Constructing a GUI in Java}
Initially, a Java simulation was developed, which aimed to construct a pixel grid which could be used to represent a wave-field. Objects representing a given data point and a "frame" of these data points were constructed, and populated with amplitudes using (\ref{equ:basicHeight}). These amplitudes were then displayed in 2D by assigning them to an opacity scale, with 100\% opacity representing the maximum possible height and 100\% transparency representing the minimum possible height. For a 40,000 pixel frame running over 10 seconds, this process took approximately 9 seconds, but the animation process after this ran in real time. Figure \ref{fig:javaBasicHeight} shows a still image of this GUI taken when the "droplet" was at a minimum height of $-h_0$. This animation was a success, but at higher resolutions, latency when drawing the pixels to the screen caused it to lag, suggesting a need to either run multiple drawing tasks in parallel, or to display the droplet motion in an alternative way.

\begin{figure}
    \centering
    \includegraphics[width=\textwidth]{simulation/javaMaxHeight.png}
    \caption{A basic Java GUI, here showing the droplet at its minimum height of $-h_0$}
    \label{fig:javaBasicHeight}
\end{figure}

\subsection{Probabilistic prediction of position}
Having demonstrated the possibility of displaying wave-functions in Java, displaying an object representing a droplet was the next task. Here, the assumption was that the wave (described by Equation \ref{equ:probWaveEqn}) represented a probability density function. Therefore, for a given (and ideally infinitesimal) pixel n of area dA, the probability $P_n$ of finding the particle at pixel $n$ was calculated from that function. The height ($z_n$) of each pixel within that section was calculated, with ${P_n}/{z_n}$ representing the probability of the particle being at that pixel. The sum of probabilities for this section, $Z=\sum_n{P_n}$, defined the normalisation value of the thread. A random number generator then generated a number R such that $0\leq R \leq 1$, which was multiplied by $Z$ to give a relative random number. The program then looped back over all the pixels in the section, and repeatedly subtracts $P_n$ from $RZ$ until $RZ<0$. The first pixel where this condition is satisfied is determined to be the location of the droplet. This process could then be multi-threaded to improve computational efficiency.

The application of this to our project was that once the droplet is found at a given pixel, the distance between that pixel and the next pixel representing the location of the droplet is used to calculate the velocity $v$ of the particle, assuming the droplet moves to the new pixel in the space of one period $T=1/f$. The wave-function was updated with the new velocity, once a Lorentz transform was accounted for. This whole process is repeated to find the trajectory of the pixel.

\begin{figure}
\centering
\includegraphics[width=\textwidth]{simulation/probabiltyPosition.png}
\caption{Results of the probabilistic prediction of the droplet position, where each dot represents the droplet being present at this point}
\label{fig:probabilisticPrediction}
\end{figure}

\begin{equation}
    h = -\cos(\omega t) J_0(\frac{\omega}{c} |\vec{r_f}-\vec{r_i}|)
    \label{equ:probWaveEqn}
\end{equation}

Although this process was successful, all it ended up proving was that a random position generator works. It does not accurately simulate the position of droplets created in our experiment. Therefore, we proceeded to calculate the equations of motion and the wave-function of the droplet at each point.

\todo{Finalise writeup of MATLAB work}
\subsection{Rapid prototyping in MATLAB}
In parallel to this, MATLAB was chosen to test implementations of mathematical concepts found in our research, as it has a wide variety of built in libraries, such as for graphing software and advanced mathematical processes. Initially, equations of motion for the droplet were taken from \cite{oza2013trajectory}, where (\ref{equ:MATLABPilotWave}) defines the wave-field of the particle and (\ref{equ:MATLABWaveHeight}) defines the bath height over time. To implement these, a grid object was constructed and populated with x-y coordinates. A point in the middle of the grid was then selected as the starting position for a droplet, which was used as the centre of modelling for the first interaction.

\begin{equation}
m \vec{x}'' + D\vec{x}' + k\vec{x} = -mg\nabla h(\vec{x},t)
\label{equ:MATLABPilotWave})
\end{equation}

\begin{equation}
h(\mathbf{x},t) = \sum_{n=-\infty}^{\floor{t/T_F}} A \mathbf{J}_0(k_F |\mathbf{x}-\mathbf{x}_p(nT_F)|) e^{-(t-nT_F)/(T_F Me)}
\label{equ:MATLABWaveHeight}
\end{equation}

Once a centre had been chosen, the wave-field generated by the droplet at that point was calculated at time $t=0$. Assuming that the droplet bounces in phase with the forcing frequency, the droplet next interacts with the bath at time $t=T_F$, where $T_F = 2/\omega$, with $\omega$ representing the forcing frequency of oscillation. The evolution of the wave-field during this time period was calculated from the Bessel function $\mathbf{J}_0$, and so the droplet experiences a quasi-instantaneous acceleration proportional to the gradient of the wave-field at the droplet's position. A wave-field is generated at the droplet's new position following this acceleration. This process is repeated, with the $N$ most recent wave-fields added to the current wave-field, where $N$ represents the number of recent impacts used to calculate the wave-field. Each iteration of this process was represented with a frame in the animation, which contained a colour-coded $(x,y,z,t)$ point for the entire grid, as shown in Figure \ref{fig:MATLABMaths}.
\begin{figure}
\centering
\includegraphics[width=\textwidth]{simulation/matlab.png}
\caption{The wave-field of a single droplet after multiple iterations, with a box size of 200x200 pixels and $N$=16. $N$ was chosen as the machine used was capable of 8 threads and each impact is calculated in parallel}
\label{fig:MATLABMaths}
\end{figure}

\subsection{Modelling assumptions and simulation mechanics}

The following assumptions were made during the simulation: 

\begin{enumerate}
\item The particle and surface waves produced are oscillating in phase with each other.
\item The particle and surface only interact within a small time frame $T_i$, after the particles' lowest point.
\item The average force exerted by the particle over $T_i$ is given by an effective wave force $F_b$. $F_b$ depends on material parameters and the mass of the particle, and tt has a maximum magnitude equal to the weight of the particle.
\item The effect of the Lorentz transform is negligible.
\item The impulse applied to the particle in the vertical direction perpendicular to the surface is assumed to be negligible and the particle continues oscillating vertically at the same frequency and amplitude. 
\item The particle experiences no damping force
\item The overall wave equation $h(\vec{r} , t)$ is dominated by the waves from the last $N$ bounces only. Waves formed longer than $N$ bounces ago are removed from the overall wave equation. This simplifies the computation of the infinite sum within the overall wave equation. In the Java simulation, by integrating multi-threading methods into our calculation, we were able to efficiently perform calculations with $N = 300$.
\end{enumerate}

Following these modelling assumptions, the parameters of the system were set as follows \cite{dotwave}:

\begin{itemize}
\item Particle mass, $m_p = 2.6\times10^{-7}$ kg
\item Driving Frequency $\omega= 80$ Hz
\item Period of particle, $T_f = \frac{2}{\omega}$ s
\item Gravitational acceleration, $g = -9.81$ m/s
\item Effective force, $F_b = 1.3174\times10^{-6}$ N
\item Wave number, $k = 1250$
\item Amplitude, A = $\frac{F_b}{m_pkg}$ m
\item N = 300
\end{itemize}


From the assumptions made above, it can be shown that the change in velocity of the particle in the direction parallel to the surface is given by:
\begin{equation} \Delta \vec{v} = \frac{T_i  F_b}{m_p} \times \frac{dh(\vec{x} , t)}{d\vec{x}}\end{equation}
Where $m_p$ is the mass of the particle and the last term is the vector gradient of the wave.


It was difficult to strike a balance between an acceptable data generation time and simulation accuracy. With simulation values used in the image above, each frame takes over 10 minutes to generate. The largest factor in the total processing time is the number $N$ of previous impacts stored. In the physical situation, the decay of each wave generated at an impact would be dependent only on the $Me$ parameter. \cite{couder11} In numerical simulation, only a limited number of recent impacts can be stored within the memory but ideally the value of $N$ should be chosen such that the contribution of an impact is nearly negligible when it is removed from memory. However, due to computational limitations, only the 16 most recent impacts were used in the calculation of the wave-field. This also precludes the simulation of multiple droplets, as each droplet would result in an additional $N$ impacts being stored.

The issues encountered in terms of processing speed could be partially addressed by porting the mathematical model developed into Java.  While it was expected that this would enable more efficient data generation, the realistic improvement in speed that could be offered would not be able to entirely negate these problems.
It was clear at this point that, regardless of potential optimisation, the goal of having a interactive simulation was not within the scope of the project. It was subsequently  decided to focus on using the previously developed Java graphics system with the aforementioned prototype  to create a number of pre-rendered videos.

\subsection{Integrating Java and MATLAB}

Due to calculation and graphical issues in  MATLAB and Java respectively, a possible solution was the execution of Java classes from a MATLAB script. This would allow the calculations to be carried out in Java, outputted to a csv or text file, and then read by the initial MATLAB script for demonstration. This was implemented using the command() function in MATLAB to navigate to the source folder, compile and then run the necessary classes. Due to time constraints, it was not possible to complete this. The main issue encountered was the use of external .jar files (to implement the Bessel function) not allowing the java files to compile. Various unsuccessful solutions were attempted,so it was decided to move on.



subsection{Java}

Importing the above model from MATLAB into Java was decided as the best course of action. The equations used and assumptions made were identical, but there was a slight difference in the manner which the gradient was calculated.

As opposed to using a built in function, in 1-D, the gradient at a point $x = x_0$ is given by:
\begin{equation} \frac{dy}{dx}\Bigr|_{x_0} = \lim_{x\to0} \frac{y(x+\delta x)-y(x-\delta x)}{2\delta x}\end{equation}

Using a Taylor expansion about $x_0$, it can be shown that for $\delta x\neq 0$:
\begin{equation} \frac{dy}{dx}\Bigr|_{x_0} = \frac{y(x+\delta x) - y(x-\delta x)}{2\delta x} - O(\delta x)\end{equation}
Where $O(\delta x)$ is the residual given by:
\begin{equation}O(\delta x) = \sum_{n=2}^{\infty} \left( \frac{y^{(n)}(x_0)}{n!\times (2\delta x)} \left[ \left( x+\delta x -x_0 \right)^n - \left(x-\delta x -x_0 \right)^n \right] \right)\end{equation}

The 2-D vector gradient of $h(\vec{r} , t)$ was determined by first finding the 1-D gradient in the x-direction by using $\delta \vec{r} = (\delta x,0)$, followed by that in the y-direction by using $\delta \vec{r} = (0,\delta y)$. The 1-D gradients in each direction corresponds to the component of the 2-D vector gradient in their respective directions. 

In our computational model, $|\delta \vec{r}| = 1\times 10^{-18}$ was used. The resolution of the Bessel functions used prevented the use of any value smaller than this. Attempts using $|\delta \vec{r}| = 1\times 10^{-19}$ resulted in gradients calculated to either have magnitude 0 or magnitude $\approx 30$, which does not match the shape of the first order Bessel function.

The wave velocity was calculated to be $\approx 0.201$. The perturbation velocity was on the order of 0.001, corresponding to $\gamma = 1.00001$. For simplicity of simulation, the Lorentz effects were assumed to be negligible, and left out of calculation.

The simulation was produced by creating .png files in Java, then stitching them into a video using VirtualDub, which could then be presented to others.


\subsection{Simulation in the high- and low-memory regimes}

The goal of the first simulation was to show that walking occurs as a result of the high memory regime. To test this, two simulations were run, one at Me = 150 and one at Me = 15. In both simulations, N = 300 was used. The particle was allowed to first bounce on the spot to build up 300 waves in the overall wave equation, corresponding to t = 7.5s. The particle was then perturbed by spontaneously changing its velocity to $\vec{v} = (0.0005,0)$. The results of the high and the low memory regime simulations are shown in Figure \ref{fig:memory}. 

\begin{figure}
	\centering
	\begin{subfigure}{\textwidth}
		\includegraphics[width=\textwidth]{simulation/highmemory/displacement.png}
		\caption{Graph of displacement in the x-direction over time. $Me=150$ represents the high memory regime, whereas $Me=15$ represents the low memory regime}
		\label{fig:mem:displacement}
	\end{subfigure}
	
	\begin{subfigure}{\textwidth}
		\includegraphics[width=\textwidth]{simulation/highmemory/velocity.png}
		\caption{In the high-memory regime, the particle velocity fluctuates chaotically before entering a stable state $\approx 11$s after perturbation. In the low-memory regime, the particle has negligible velocity}
		\label{fig:mem:velocity}.
	\end{subfigure}
	
	\begin{subfigure}{0.475\textwidth}
		\includegraphics[width=\textwidth]{simulation/highmemory/wavefield75.png}
		\caption{The overall surface wave after $7.5$s; the wave-field has a similar shape to a 2-D harmonic potential well}
		\label{fig:mem:wavefield75}
	\end{subfigure}
	\hfill
	\begin{subfigure}{0.475\textwidth}
		\includegraphics[width=\textwidth]{simulation/highmemory/wavefield145.png}
		\caption{The wave-field at $14.5$s, just prior to the particle breaking through the potential barrier and travelling}
		\label{fig:mem:wavefield145}
	\end{subfigure}
	
	\begin{subfigure}{0.475\textwidth}
		\includegraphics[width=\textwidth]{simulation/highmemory/wavefield165.png}
		\caption{Following multiple interactions with wave peaks, the particle decelerates and reforms a potential well}
		\label{fig:mem:wavefield165}
	\end{subfigure}
	\hfill
	\begin{subfigure}{0.475\textwidth}
		\includegraphics[width=\textwidth]{simulation/highmemory/wavefield195.png}
		\caption{The stable wave-field at $19.5$s, now displaced by $0.01m$.}
		\label{fig:mem:wavefield195}.
	\end{subfigure}
\caption{Results of the simulation in the high- and low-memory regime at different time periods}
\label{fig:memory}
\end{figure}

We see from Figure \ref{fig:mem:velocity} that the particle velocity is chaotic. This causes the particle to oscillate around two centres of displacement, initially the start position, and later about the point $(0.01,0)$. We also observe in Figure \ref{fig:mem:wavefield75} that, prior to perturbation, the wave-field is similar in shape to a 2-D harmonic potential well, of width $\lambda/2$, and centred about the initial position of the particle. Upon perturbation, the particle oscillates, as expected for such a system. However, with each impact at positions away from the start position, new waveforms are added to the overall wave-function at these points. The waves formed at this point superpose over the 2-D harmonic potential well, flattening it along the x-axis. After approximately 7s, the peaks at $\lambda/2$ away from the start position along the x-axis were low enough that the particle could leave the well as shown in Figure \ref{fig:mem:wavefield145}.

The overall surface wave at 7.5s is shown in Figure \ref{fig:mem:wavefield75}. The wavefield is observed to be similar in shape to the 2-D harmonic potential well within a half wavelength about the particles' start position. At this point, the particle is perturbed. As expected for a particle oscillating in a 2-D harmonic potential, the particle enters oscillatory motion. However, with each impact at positions away from the start position, new waveforms are added to the overall wavefunction at these points. The wave forms at this point superpose over the 2-D harmonic potential well flattening it along the x-axis. After approximately 7s, the peaks at half wavelength away from the start position along the x-axis were low enough that the particle could traverse it and leave the well about the start position. The overall surface wave at this point is shown in Figure \ref{fig:mem:wavefield145}. As the particle carries on in its trajectory away from the start position, it interacts with other smaller peaks, causing it to slow down. As the particle slows down, the most recently formed waves are centered closer and closer together, eventually reforming yet another potential well about the point (0.01,0). The overall surface wave at this point is shown in Figure \ref{fig:mem:wavefield165}. The particle then starts oscillating about this point for approximately 4s before the same mechanisms that allowed it to break through the first potential well occurs, and the particle leaves the new formed well. Following this the particle enters a steady state and continues travelling with a stable velocity of (-0.043,0) $ms^{-1}$. The particle has thus started ``walking".

The high memory regime simulation was repeated for perturbation velocity with magnitude between 0.0001 $ms^{-1}$ and 0.0020 $ms^{-1}$ at intervals of 0.0001 $ms^{-1}$, to ensure that the above result of achieving walking was not due to specifically selected variables. In all cases the particle entered a steady ``walking" state with velocities ranging form magnitude 0.00859 to 0.102. The final steady state velocities of each particle and their corresponding perturbation velocity can found in \ref{fig:pertVfinal}. No observable trend was found relating the steady ``walking" state velocity and the perturbation velocity.
\begin{figure}
    \centering
    \includegraphics[width=\textwidth]{simulation/figpert.png}
    \caption{Results of the simulations in the high memory regime at velocities ranging from 0.0001 $ms^{-1}$ and 0.002 $ms^{-1}$}
    \label{fig:pertVfinal}
\end{figure}

The above results strongly suggest that a particle entering the walking state depends on $Me$. Simulations were next conducted for $Me$ between 15 and 150 at intervals of 0.1. Perturbation velocity was kept constant at (0.0005,0) $ms^{-1}$ in all simulations. The following conditions were set to determined if the particle has entered a walking state:

\begin{enumerate}
    \item $\Delta\vec{v} = 0$ for the past 20 time steps.
    \item The particle is at least 1 wavelength away from where it was initiated.
\end{enumerate}

If the particle in a simulation was detected to be in a walking state, the simulation would output the walking state velocity and the time at which "walking" was achieved. Each simulation was allowed to run for 10,000 time steps, corresponding to 250s in real time. Any particle yet to achieve walking within this time frame was assumed to not be capable of entering the walking state. Visualisations of this data are shown as graphs in Figure \ref{fig:varmem}. 

As shown in Figure \ref{fig:varmem:walkingstate}, there is no clear threshold $Me$ beyond which the particle will enter a walking state. From the plot it is observed that walking will not occur for $Me < 50$ but will occur for $Me > 80$. The region in between, given by $50\leq Me\leq80$, however, appears to be a grey area where whether a given Me leads to a walking state seems to follow some distribution. To determine the shape of this distribution a moving average of the previous 20 and next 20 points was plotted against $Me$. By observation, the plot here appears to be in the shape of a sigmoid curve. Using Origin Pro 2017s' non-linear curve fit function, the plot in was fitted to a Boltzmann sigmoid function with the top and bottom value fixed at 1 and 0 respectively. The Boltzmann sigmoid is given by \ref{equ:BoltzmanSigmoid}.

\begin{equation}
y = (bottom) - \frac{(top)-(bottom)}{1+exp \frac{\left(x_0 - x \right)}{dx}}
\label{equ:BoltzmanSigmoid}
\end{equation}

Here $x_0$ is the 50\% threshold found to be $64.54\pm0.04$.

The fit was found to have a R-squared value of 0.9974 and a reduced chi squared value of $5.33 \times 10^{-4}$ suggesting a decent fit. The plot of the residual of the fit against $Me$, as shown in Figure \ref{fig:varmem:residualplot}, interestingly seems to be in the form of a wave packet. 

Figure \ref{fig:varmem:timewalking} shows the relation between time taken to enter the walking state and $Me$. It is observed that the time taken has a wider spread for lower Me, with a minimum that remains constant as Me increases. The minimum has a value of approximately $3\pm2$. This corresponds to the time required to flatten the sides of the wave as described previously, thus allowing the particle to pass through it and leave the potential well around it.

The plot of the x-component of the walking velocity against $Me$ is shown in Figure \ref{fig:varmem:walkingvel}. Since the particle was perturbed in the x-direction only, the y-component in all cases is 0. It is observed that approximately half of the simulations produced a walking velocity in the positive x-direction (the direction of perturbation), while the remaining half in the negative x-direction. 



\begin{figure}
	\centering
	\begin{subfigure}{\textwidth}
		\includegraphics[width=\textwidth]{simulation/varmemory/walkingstate.png}
		\caption{Binary plot showing whether a particle was found to have entered a walking state within 10,000 time steps with 1 corresponding to yes and 0 to no against $Me$. The moving average is plotted and fitted to a sigmoid curve}
		\label{fig:varmem:walkingstate}
	\end{subfigure}
	 \begin{subfigure}{\textwidth}
		\includegraphics[width=\textwidth]{simulation/varmemory/residuals.png}
		\caption{Residual plot of the sigmoid fit of \ref{fig:varmem:walkingstate}}
		\label{fig:varmem:residualplot}
	\end{subfigure}
	\begin{subfigure}{\textwidth}
		\includegraphics[width=\textwidth]{simulation/varmemory/meWalkingState.png}
		\caption{Plot of time taken for the particle to enter walking state against $Me$ used in the simulation}
		\label{fig:varmem:timewalking}
	\end{subfigure}
	
	\begin{subfigure}{\textwidth}
		\includegraphics[width=\textwidth]{simulation/varmemory/velocityWalkingState.png}
		\caption{Plot of walking velocity of the particle against $Me$ used in the simulation}
		\label{fig:varmem:walkingvel}
	\end{subfigure}
\caption{Analysis of the probability of a particle entering the walking state at different values of $Me$}
\label{fig:varmem}
\end{figure}

\subsection{Preliminary tests on predicted phenomena}
The theoretical models used in our simulations also made other predictions as to the behaviour expected. To further provide supporting evidence for the validity of our  assumptions, we attempted to demonstrate some of these predictions. Due to time constraints, we only managed to run preliminary tests demonstrating oscillating droplets and wall repulsion.

\subsubsection{Oscillating droplets}

As predicted, \cite{brady2014bouncing} bouncing droplets experience an attractive force when they are bouncing in phase. Experimental evidence provided by the same team suggests that this force follows an inverse square relation similar to gravity and electromagnetism. A preliminary test was conducted by performing a simulation that initiated two droplets, separated along the x-axis by 0.004m. The droplets were initiated with velocities in the y-direction with magnitude 0.0003 $ms^{-1}$ in opposite directions. This resulted in the droplets having an angular orbital momentum between themselves.

In the simulation performed, it was observed that the droplets initially entered some form of orbital motion, as expected from an attractive inverse square force. However, orbital motion was only maintained for an equivalent real time of approximately 6s before the system destabilises. The droplets were then observed to enter walking states in opposite directions. Due to time constraints, further detailed analysis on the motion of the droplets were not performed. 

\subsubsection{Wall repulsion}

Wall repulsion was simulated by initiating a droplet displaced along the positive x-direction. A wall was positioned along the y-axis. By assuming the wall is completely reflective, an incoming wave should reflect off the wall out of phase, and in the direction given by the law of reflection. To simulate this, inspiration was taken from the method of mirror charges in electrostatics. Each time a waveform was added to the overall wave function, a “mirror” waveform was also created. The “mirror” waveform added was out of phase with the droplet, and initiated at the position given by the reflection of the droplet in the y-axis.

The simulation demonstrated that the droplet accelerates away from the wall, as expected. Due to time constraints, no detailed analysis was performed on the motion of the particle so the trajectory of repulsion was not quantified. The effects of changing variables was also not explored.


\subsection{Conclusion}

The original aims of the simulation were to construct an interactive software package that could simulate and visualise the motion of droplets bouncing on a liquid surface. These aims were partially met: a non-interactive simulation was developed, whereby images from the software output could later be reconstructed into a video. However, due to computational limitations it was not possible to run this process in real time. This posed a significant barrier to the use of the simulation as a real-time educational aid, although the videos were still useful to illustrate the concepts to students in a classroom.

As work continued with the simulation, and the team developed a more detailed understanding of the theory behind droplet motion, additional goals were added to the project. This included determining the effect of changing simulation parameters such as the memory coefficient $Me$, and allowing the end-user to vary those without needing to understand the source code. Again, these were partially met: detailed analysis of the effect of $Me$ has been offered, an insight which would not be possible within the physical system. However, simulation parameters are still hard-coded, although this could change with relative ease in the future.

While a more quantitative analysis of boundary conditions and multiple droplets would have been desirable, the preliminary results obtained agreed with qualitative predictions. This suggests the simulation's behaviour is consistent with theory, but without quantitative comparisons, the accuracy of this could not be verified. This meant that the subsidiary goal of representing quantum effects, such as double-slit diffraction, was not met. To provide a more accurate model, several other factors that were assumed negligible could be accounted for, such as adding a parameter to represent the inelasticity of collisions between the droplet and the surface.