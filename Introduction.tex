\section{Introduction}
An age-old and much investigated field of research in science has been into the nature of light. For centuries, scientists have endeavoured to understand its nature and fundamental properties. At the turn of the 20th century, light was widely believed to behave as a wave, but the discovery of phenomena such as the photoelectric effect called this theory into question. Eventually, it was agreed on that light exhibited both wave-like and particle like behaviour, a phenomenon known as wave-particle duality. The interpretation of this quantum mechanical behaviour, however, is still open to debate. The commonly accepted interpretation, known as the Copenhagen Interpretation, was first proposed by Niels Bohr and Werner Heisenberg in 1927. It pushed forward the idea that a physical system would only have definite properties upon measurement; prior to measurement it was only possible to measure probabilities of obtaining a given measurement. An alternative interpretation, the De Broglie – Bohm theory, was pushed forwards in 1952. It is deterministic in nature; it claims that even prior to observation there exists a driving equation or 'pilot wave' whose properties determine measurements. In 2005, research by \textit{Couder et al.} \cite{couder} suggested that droplets bouncing on a surface could display quantum mechanical behaviour as proposed by the pilot wave theory. Indeed this macro system was a remarkably good analogy  for the pilot wave theory; waves induced by a driving droplet quite closely match a driving wave-equation determining the results of measurements. 
\subsection{Aims}
To this end, the purpose of this project is to develop a cost-effective table top version of the experiment used by \textit{Couder et al.},demonstrating the various different behaviours of bouncing droplets. Such motion can include simple oscillations, motion parallel to the liquid surface and double slit diffraction of droplet generated waves. It is intended to be paired with a computer based visualisation that can display effects outside the bounds of this apparatus. The overarching goal will be to develop a prototype that can be used for outreach purposes, allowing  students to the grasp the non-intuitive nature of quantum mechanics. 

\subsection{Objectives}
These above aims raise the following core project objectives: 
\begin{enumerate}
   \item Development of table top apparatus: 
   \begin{enumerate}
     \item To create a working first prototype capable of displaying basic  motion, such as bouncing and walking droplets
     \item To develop an effective method by which the rapid motion of droplet oscillations can be recorded.
     \item To further improve the prototype, and demonstrate multiple droplet motion (such as droplets repelling and coalescing) and phenomena such as double slit diffraction
    \end{enumerate}
   \item Development of an in-depth visualisation simulation:
   \begin{enumerate}
     \item To identify and implement the mathematical equations necessary for displaying single droplet motion in Java. 
     \item To extend the capabilities of the program in order to cover aspects of quantum mechanics not displayable by the experimental apparatus. Such effects may include double slit diffraction and 'tunneling' effects, where the droplet escapes over the edge of the apparatus. 
    \end{enumerate}
    \item Creating an iteration of the initial prototype usable in college/undergraduate courses as a demonstration tool.  
   \begin{enumerate}
     \item To create a standalone version of the simulation program.
     \item To create a simple to use version of the experimental apparatus, for third-party users. 
     \item To write a business plan for implementation in future.
     \item To explore the possibility of testing the apparatus with undergraduates and A-level students. 
    \end{enumerate}
\end{enumerate}
