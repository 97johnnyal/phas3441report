\noindent 
\section{Minutes 11/1/18}\label{app:11-1}

\noindent Meeting Time: 4pm Thursday 11th Jan

\noindent Meeting Location: Massey Group Study Pod, 3rd Floor, Science Library
\\\\
\noindent \textbf{\underbar{Attendance}}

\noindent Chair: Chong Keat Gea CKG

\noindent Vice-Chair: Kelvin Fang KF

\noindent Secretary: Johnny Allain-Labon JAL

\noindent Treasurer: Mohit Motwani MM

\noindent Georges Ajaka GA

\noindent Course Coordinator Point of Contact: Steven Vuong SV

\noindent Benji Berczi BB

\noindent Alex Stock AS

\noindent Prof. Ryan Nichol RN

\noindent 

\noindent Present: CKG KF JAL MM GA SV BB AS RN

\noindent 
\\
\begin{enumerate}
\item  \textbf{Project Outline}

\begin{enumerate}
\item Existing equipment: none, constructing prototype from scratch

\item  Prototype - start as simple as possible e.g. petri dish on loudspeaker e.g. \url{https://www.youtube.com/watch?v=WIyTZDHuarQ}.  Move on to more complex behaviour if possible

\item  Simulation Programme - want to demonstrate quantum behaviours of the bouncing oil drops if experimental verification is problematic + add missing features of experiment.\\
\end{enumerate}

\item  \textbf{Aims \& Objectives}

\begin{enumerate}
\item Construct prototype:

\begin{enumerate}
\item  Drop bouncing

\item  Stable drop bouncing over several seconds

\item  Getting the drops to walk

\item  Drop interaction with boundaries

\item  Two or more drops

\item  Drops orbiting

\item  Double slit interference

\item  Tunnelling
\end{enumerate}

\item  Simulation: demonstrate same effects on a computer

\begin{enumerate}
\item  Start with 2D plot

\item  Look to animate with 3D movement
\end{enumerate}

\item  Aim: outreach tool for demonstrating quantum effects on real life scale

\begin{enumerate}
\item  Ideally interactive i.e. can add new droplets

\item  Replicable by teachers
\end{enumerate}

\item  Stretch aim: video demonstrating + different containers/parameters e.g. frequency - bounce to a song?\\
\end{enumerate}

\item  \textbf{Assessment Criteria}

\begin{enumerate}
\item Working prototype

\item  Working simulation

\item  Reporting\\
\end{enumerate}

\item  \textbf{Deadlines}

\begin{enumerate}
\item 14/1 - Research deadlines - need understanding of scope of project i.e. read papers. Including summary of papers for presentation in formal report. Maintain file ``notes and background reading''

\item  18/1 4pm - Plan for Prototype + Outline for simulation

\item  12/2 - READING WEEK - Basic simulation + prototype completed

\item  Reading Week - report to other group

\item  7/3 - prelim deadline for poster being finalised for printing

\item  16/3 5pm - FINAL REPORT due + critical self-assessment

\item  21/3 - poster presentation\\
\end{enumerate}

\item  \textbf{Areas of Responsibility}

\begin{enumerate}
\item General Time Management - JAL

\item  Prototype Weds 17/1 6pm brainstorm

\begin{enumerate}
\item  SV Lead

\item  KF

\item  BB

\item  MM

\item  CKG
\end{enumerate}

\item  Simulation - written in Python - Weds 17/1 6pm brainstorm

\begin{enumerate}
\item  GA Lead

\item  AS

\item  JAL
\end{enumerate}

\item  Written Reporting - written as we go - MM\\
\end{enumerate}

\item  \textbf{Communications Plan}

\begin{enumerate}
\item Google Drive for documents

\item  ShareLatex for written reports

\item  Facebook group for updates, requests for help, anything permanent

\item  FB chat for real time but ephemeral communication\\
\end{enumerate}

\item  \textbf{Future Meetings}

\begin{enumerate}
\item Thursdays 4pm - regular slot

\item  Thursday 18/1 13:30-14:00 subgroup meeting

\item  Thursday 18/1 16:00 - KF to book

\item  RK availability - to be emailed

\item  Lab use - Derick will accommodate people available, 9-5 lab hours with lunch at 1. Other spaces available for building things - Institute of Making - CKG has one
\end{enumerate}
\end{enumerate}


