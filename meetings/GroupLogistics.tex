\section{Group Organisation (MM)}
\subsection{Assigned Roles}
Central to any successful project is effective, efficient division of labour, in order to ensure that all necessary tasks are accomplished on time, and to a high standard. In order to accomplish this, it was decided that the project, having two clear avenues to follow, required the creation of two subgroups: an experimental group and a simulation group. The former were to handle the creation of the apparatus, experimentation and preliminary outreach efforts, whilst the latter were to handle the creation of a simulation that could effectively display these physical phenomena. Of course, due to the varied nature of the project and constantly fluctuating workloads, several changes were made to personnel roles over the course of the project. Therefore, a brief outline of each group member's responsibilities is given below:

\begin{itemize}
    \item Georges Ajaka (GA): 
\end{itemize}

\todo{Fill out own section about responsibilities in project}

\subsection{Budget}
To carry out this project, we received an initial sum of \pounds200. This was to include all manufacturing costs, alongside the cost of poster printing. It was an ambitious limit, considering previous investigations of a similar similar were carried out in full-fledged research laboratories, with the corresponding resources. Nevertheless, some of the initial objectives were accomplished, with a substantial sum being left over at the summation of \todo{insert final sum left here before final print} of the project. The full expenditure details are given in Table \ref{table:budget}. 

\bigskip                    
\begin{table}
\centering
\begin{tabular}{|c|c|c|}
\hline
&\textbf{Income}& \textbf{Expenditure}\\
\hline
\textbf{Initial budget}& \pounds200&  \\
\textbf{Plywood}& &\pounds3.00  \\
\textbf{ELEGIANT amplifier}& &\pounds12.99  \\
\textbf{Foldback clips}& &\pounds3.00  \\
\textbf{Photron high speed camera rental x2}& &\pounds50.00  \\
\textbf{Outreach travel expenses}& &\pounds24.00  \\
\textbf{Sharelatex account}& &\pounds7.20  \\
\textbf{Draft report print costs}& &\pounds1.85  \\
\textbf{LED lights}& &\pounds6.99  \\
\textbf{Final report and binding}& &\pounds TBC  \\
\textbf{Amplifier power supply}& &\pounds6.99  \\
\textbf{Poster}& &\pounds 40.00  \\
\hline
\multicolumn{3}{|c|}{\textbf{Total expenditure \pounds161.86}}\\
\hline
\multicolumn{3}{|c|}{\textbf{Sum remaining \pounds38.14}}\\
\hline
\end{tabular}
\caption{A summary of the expenses endured over the course of this project. Values for both the money remaining and total spent are given.}
\label{table:budget}
\end{table}
\todo{REPORT PRINTING COSTS}

\subsection{Communication and Administration}
In order to keep both teams updated on group progress overall, and to be able to appropriate specific skills of group members, frequent meetings were in order. Weekly meetings were held, occasionally with the board member present, in order to track progress on a week-by-week basis, and to keep on top of approaching deadlines. Prior to each meeting, the group chair produced an agenda, which acted as a framework for the meeting. Meetings were recorded via minutes, made by the group secretary. These were published online, to give all members a clear outline of responsibilities for the forthcoming week. As this project required group access to a large amount of shared material, a Google Drive folder was set up and organised into clear sub-folders. This was made available to all group members from the start. Group and sub-group communication was handled via Facebook Messenger, with important links pinned on a Facebook group, in order to ensure that no important links were lost in chat. 

Sub groups met once a week to discuss their work further. The experimental team, before splitting to form an educational team, met more frequently to carry out lab work. The simulation team coordinated their efforts through GitHub, which allowed them to sync their work, while also acting as a fail-safe in case of a loss of local files. The prototype group used Google drive to share resources. 
The report was written using ShareLaTeX, an online LaTeX processing tool that allowed for collaborative access, for changes to be tracked and for comments to be made. A template was utilised to make the poster on ShareLaTeX as well. 