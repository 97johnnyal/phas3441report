\section{Group Organisation (MM)}
\subsection{Assigned Roles}
Central to any successful project is effective, efficient division of labour, in order to ensure that all necessary tasks are accomplished on time, and to a high standard. In order to accomplish this, it was decided that the project, having two clear avenues to follow, required the creation of two subgroups: an experimental group and a simulation group. The former were to handle the creation of the apparatus, experimentation and preliminary outreach efforts, whilst the latter were to handle the creation of a simulation that could effectively display these physical phenomena. Of course, due to the varied nature of the project and constantly fluctuating workloads, several changes were made to personnel roles over the course of the project. Therefore, a brief outline of each group member's responsibilities is given as follows:

\textbf{Mohit Motwani} (MM) initially took responsibility for the final written report and has seen the whole report through to the end, writing both the intro and theory. Johnny also jumped aboard at a later stage to assist with the editing. Alongside this, Mohit also worked on the experimental prototype during the first half of term whilst equipment was being gathered. Specifically, he researched into slow-motion photography for tracking droplet motion as well as recorded meeting minutes. Following the subdivision, Mohit liaised with Mark Fuller and Johnny to create a lesson plan and worked with Benji to outline the objectives for outreach. The lesson plan was eventually executed at Chesham Grammar School in a pilot outreach lesson with Johnny and Steven.

\textbf{Johnny Allain-Labon} (JAL) took responsibility as group secretary, communicating with the board member and the group as well as typing up minutes. Having done PHAS3459, Johnny investigated the animation aspect of wave-functions in python and Java, coding basic generation of frames used by the simulation group as well as helping report on work accomplished. Following this, Johnny co-created and eventually delivered a lesson plan for Chesham Grammar School, leading up with the results of the outreach effort. Finally Johnny held joint responsibility with Mohit in writing and editing the written report and structured and made the typeset for the poster.

\textbf{Chong Keat Gea} (CKG) took responsibility as chair, in doing so produced meeting agendas, managed group dynamics and tracked time management ensuring that simulation and prototype groups were progressing at a steady rate. Additionally Gea helped distribute workload so team members contributed equally by conducting ‘health checkups’ at every meeting, delegating additional roles to team members who felt they were not contributing enough. Gea also interacted with staff at the Institute of Making and developed the prototype plywood housing, also assisting with soldering and wiring of the initial prototype. Upon moving to the simulation team, Gea developed a random point generator from a probability density function, using the math in the Matlab file in Java to perform simulation tests and analysis with multiple droplet simulations.

\textbf{Steven Vuong} (SV) initially wrote a summary of the theory and research material provided as a prompt by the board member before taking responsibility as lead of prototype group. Initially with Gea, Benji, Kelvin and Mohit, Steven co-created the first prototype and managed to obtain access to recording equipment as well as other devices and expertise from various other UCL departments. Eventually Steven and Kelvin in the prototype sub-group worked in labs to make and record observations into bouncing droplets and wrote transcripts of lab sessions throughout. Steven also went into Chesham Grammar with Mohit and Johnny to aid in outreach and helped to gauge feedback for improvement before finally assisting in writing the experimental section of the final report.

\textbf{Georges Ajaka} (GA) took lead as simulation group, organising simulation group meetings and reported progress to the rest of the group in weekly meetings. Georges initially researched suitable theories which could be applied to simulation, with this he created an initial Java graphics interface where he made velocity dependent colouring scale for trajectory lines implemented on a random point generator. Furthermore, Georges worked on integrating MATLAB and Java code before assisting in writing the simulation section of the final report. Georges also made a final repository for all the code.

\textbf{Benji Berczi} (BB) started on the prototype group and researched equipment lists which were used to build the first working prototype which he also took part of. Benji was then responsible for writing the peer-assessment of the other group’s meeting and wrote their critical assessment. Alongside this, Benji collaborated with Mohit and took the lead in writing a business proposal aimed at winning an STFC run award in developing the educational aspect of the prototype and also explored commercialisation possibilities. Benji also worked on the final poster.

\textbf{Kelvin Fang} (KF) initially further developed research material and found papers surrounding the topic which were used by the simulation group. Kelvin also took initiative of coming up with a lab script for our entire lab script which he saw through and experimental log template which were used to record experimental observations. Kelvin also booked meeting rooms. Kelvin supplied various apparatus and worked with Steven, Gea and Benji to put together the apparatus. Eventually partnering with Steven to carry out experimentation, video capture and analysis. As Kelvin had access to Adobe AfterEffects, he conducted video processing, photo generation and motion tracking to reveal sinusoidal motion verifying pilot wave theory of our bouncing droplet. Finally Kelvin assisted in writing out the experimental methods section for the final report.

\textbf{Alex Stock} (AS) conducted research to find suitable theories which could be applied for simulation. As Alex had prior experience with MATLAB, he formulated the basis for the final model which was eventually ported into java. Alex wrote this whilst the rest of the simulation group simultaneously attempted to create a bouncing droplet into java. Alex then took responsibility for his code and followed it through, assisting in implementing his mode into java and writing the simulation section of the report. Finally, Alex assisted in writing the theory section of the final report.


\subsection{Budget}
To carry out this project, we received an initial sum of \pounds200. This was to include all manufacturing costs, alongside the cost of poster printing. It was an ambitious limit, considering previous investigations of a similar similar were carried out in full-fledged research laboratories, with the corresponding resources. Nevertheless, some of the initial objectives were accomplished, with a substantial sum being left over at the summation of \todo{insert final sum left here before final print} of the project. The full expenditure details are given in Table \ref{table:budget}. 

\bigskip                    
\begin{table}
\centering
\begin{tabular}{|c|c|c|}
\hline
&\textbf{Income}& \textbf{Expenditure}\\
\hline
\textbf{Initial budget}& \pounds200&  \\
\textbf{Plywood}& &\pounds3.00  \\
\textbf{ELEGIANT amplifier}& &\pounds12.99  \\
\textbf{Foldback clips}& &\pounds3.00  \\
\textbf{Photron high speed camera rental x2}& &\pounds50.00  \\
\textbf{Outreach travel expenses}& &\pounds24.00  \\
\textbf{Sharelatex account}& &\pounds7.20  \\
\textbf{Draft report print costs}& &\pounds1.85  \\
\textbf{LED lights}& &\pounds6.99  \\
\textbf{Final report and binding}& &\pounds TBC  \\
\textbf{Amplifier power supply}& &\pounds6.99  \\
\textbf{Poster}& &\pounds 40.00  \\
\hline
\multicolumn{3}{|c|}{\textbf{Total expenditure \pounds161.86}}\\
\hline
\multicolumn{3}{|c|}{\textbf{Sum remaining \pounds38.14}}\\
\hline
\end{tabular}
\caption{A summary of the expenses endured over the course of this project. Values for both the money remaining and total spent are given.}
\label{table:budget}
\end{table}
\todo{REPORT PRINTING COSTS}

\subsection{Communication and Administration}
In order to keep both teams updated on group progress overall, and to be able to appropriate specific skills of group members, frequent meetings were in order. Weekly meetings were held, occasionally with the board member present, in order to track progress on a week-by-week basis, and to keep on top of approaching deadlines. Prior to each meeting, the group chair produced an agenda, which acted as a framework for the meeting. Meetings were recorded via minutes, made by the group secretary. These were published online, to give all members a clear outline of responsibilities for the forthcoming week. As this project required group access to a large amount of shared material, a Google Drive folder was set up and organised into clear sub-folders. This was made available to all group members from the start. Group and sub-group communication was handled via Facebook Messenger, with important links pinned on a Facebook group, in order to ensure that no important links were lost in chat. 

Sub groups met once a week to discuss their work further. The experimental team, before splitting to form an educational team, met more frequently to carry out lab work. The simulation team coordinated their efforts through GitHub, which allowed them to sync their work, while also acting as a fail-safe in case of a loss of local files. The prototype group used Google drive to share resources. 
The report was written using ShareLaTeX, an online LaTeX processing tool that allowed for collaborative access, for changes to be tracked and for comments to be made. A template was utilised to make the poster on ShareLaTeX as well. 

\subsection{Group Dynamics}
Overall, there were no issues with group dynamics. Meetings were held frequently, with regular attendance from all members. All members contributed equally, with responsibilities being shared and taken on in a concientou 