\section{Initial Apparatus}
The initial version of this apparatus is based  off previous run experiments by \textit{Harris et al.}. It consists of a 250 W, 8 Ohms High Powered Woofer With Aramid Fibre Cone, serving as the source of vibrations, mounted in a custom made ply-wood frame. The speakers volume, and thus the amplitude of the vibrations was altered through either a a signal generator or a signal generating app on a phone. In order to ensure that vibration amplitude was sufficiently large, an amplifier[specify type] was used. This amplifier also allowed for bass and treble modification. 

A piece of plywood with a 6.5 inch diameter, that fit perfectly on the rim of the speaker.Graph paper was then cut into shape and placed on the wooden surface. The graph paper was attached to the wood surface using double sided tape. The wood surface was secured to the speaker using a combination of fold-back clips and double sided tape. A [] inch petri dish was filled with 50 cSt silicon oil. The petri dish was attached to the graph paper using double sided tape carefully placed on the edges of the dish, to ensure that the remains of tape did not coat its lower surface, which interfered with images. The droplet motion was observed using a high-speed camera capable of recording at 1,000 fps; the recording area was illuminated using two spotlights. 

\subsection{Box creation}
To be filled in with the help of Gea.\\ 


\subsection{Droplet Creation}
A key aspect of this experiment involves creating a droplet of consistent size, in order to allow for various behavioural regimes to be consistently recreated. While previous such experiments use a 'trial-and-error' approach to such matters, due to our lack or ready access to strong light sources and suitable cameras, a more consistent approach was necessary.

An initial attempt was made using a needle, or segment of wire,  dipped this rapidly into the liquid surface. As this approach was unsuccessful,  an attempt was made to flick the top of the surface with the tool tip; a semi-reliable way of creating a droplet as it required a few attempts.It was then discovered that, by rapidly plunging and  removing a segment of laser-cut plywood into the liquid surface, it was possible to create multiple droplets simultaneously [NEED WOOD IMAGE PLS]. An important point to note from \ref{regimes} was that the regimes in which walking and bouncing droplets occur was dependent on droplet diameter. Therefore, while this approach was certainly successful, refinement was necessary in order to create droplets of consistent sizes. A range of smaller droplet sizes were obtained by reverting back to the needle where necessary

\subsection{Signal Generation}
Initially, an IsoTech[TYPE NECESSARY] signal generator was chosen to produce an input signal for the speaker. However, it was soon discovered that the signal generator did not produce a signal of enough power to create bouncing droplets. Therefore an amp, and a signal generator with a gain was obtained[model no]. 

It was discovered that  the frequencies utilised (50Hz-80Hz) were low enough that the amplifier considered them to be in the bass regime. Therefore altering the treble had little effect on the vibrations produced, while changes in bass were quite significant. This was used to help increase the vertical acceleration of the droplet to the critical value necessary for walking behaviour. 

Several weeks into experimentation, it was discovered that the amp was not grounded, while the amplified signal generator was. Therefore, both devices did not have a common ground voltage, leading to an improperly functioning amp, and a reduced signal amplitude. Subsequently, it was decided to use a signal generating amp on a phone to generate the driving frequency, with an amp to boost the signal.  

\subsection{Liquid Choice}
It was discovered that liquid viscosity choice had a significant impact on droplet formation and behaviour. Initially, a 1000cSt silicon oil was used in this experiment, and the results were understandably non-conclusive. The high viscosity liquid resulted in droplets of an extremely large size, so much so that the droplets did not bounce, but established themselves on the surface for several seconds before coalescing with the surface, as predicted b[INSERT REFERENCE TO COUDER PAPER]. Therefore, a 50cSt oil was utilised instead. It is important to note that high viscosity liquids are usable in this experiment. Indeed[REFERENCE HERE] utilised such liquids to great success in their experiments. However, such liquids require vertical accelerations that the given apparatus is incapable of producing. A suitable replacement may be a vibration exciter mounted on a sturdier frame. 

A further aim of this project is to develop a commercial version of the apparatus. Doing so would require a more affordable liquid, that would be more widely available outside of a laboratory, for easy replacement. One such option is to utilise liquid soap diluted in water \ref{walker}, but these droplets have a small maximum lifetime of 18 minutes, only after a long period of operation. Secondly, vegetable oils are also an option. These options were investigated further later.

\subsection{Recording Droplet Motion}
Key to this experiment was recording the vertical motion of the droplet, which was too fast to be observed by eye. Initially, an attempt was made to use slow motion recording on phone cameras, but as the droplets oscillated at 50-80 hz, this provided 3-4 frames per droplet oscillation, which was unacceptably low for analysis purposes. Attempts were made to obtain a Phantom high speed camera, but as the rental was valued at £20 a day, and it was thus decided to use this option once the setup was refined, and all other possible options were exhausted. 

[Andrew?] Fortunately, for our initial experimentation, a PhD student from the mechanical engineering department kindly provided us with access to a 1000fps, handheld, high speed camera. He also provided us with two spotlights, vital in generating the intense illumination necessary for high speed photography. The camera and spotlight setup is as illustrated in figure []. Observing wave motion with this setup was difficult for two reasons. Firstly, the liquid was transparent, making it hard to discern where the waves formed exactly, even with slow-motion footage. This played into the second factor, which was the lack of a macro-lens for the camera. While the camera was suitable for recording the droplet to analyse its size and motion, it could not be brought to focus on the waves themselves, due to the limitations of the lens. As the camera did not allow for the mounting of such a lens, it was determined that another option, such as a phantom, would be necessary to observe wave motion.  
