\section{Experimental Log 14-2}

\textbf{Participants:} SV, KF

\textbf{Location:} Year 1 Teaching Lab, Department of Physics and Astronomy, University College London

\textbf{Start time:} 0930

\textbf{End time:} 1130

\bigskip
\textbf{Objectives}

\begin{itemize}
\item To achieve stable walking regime; have an idea of droplet size and amplitude
\item To observe orbiting pairs, repelling pairs and the maximum number of bound droplets
\item To collect data on droplet size and acceleration if possible
\end{itemize}
\bigskip


\textbf{Equipment added/modified; safety precautions}

\begin{itemize}
\item Laptop to produce wave signal

\begin{itemize}
\item syznalski.com/tone-generator/
\item tonescope.net/tone
\end{itemize}
\item 1 mm pipette
\item 3 mm pipette
\item Compressed air
\item Sound meter app
\end{itemize}
\bigskip


\textbf{What processes were carried out?}
\bigskip
\begin{center}
\begin{tabu} to 0.8\textwidth { | X[c] | X[c] | }
 \hline
 \textbf{Distance (cm)$\pm$ 1 cm} & \textbf{Sound(dB) $\pm$ 1 dB}  \\
 \hline
 10  & 80    \\
\hline
 20  & 78    \\
\hline
 30  & 77    \\
\hline
 40  & 76    \\
\hline
 50  & 76    \\
\hline
\end{tabu}
\end{center}
\bigskip
\begin{itemize}
\item Use pipette to generate droplets
\item Measure noise vs distance (problem of background)
\end{itemize}

\begin{itemize}
\item Take measurements of sound at 30 cm away. Test volume (dB) by increasing sound increments on computer

\begin{itemize}
\item Use light to illuminate which casts a shadow on the graph paper to estimate size

\begin{itemize}
\item With a 0.6 mm diameter droplet at 81 dB, we observed `walking' for a few cm
\item Other droplets of similar size would appear to have chaotic movement
\end{itemize}
\end{itemize}
\item Observe droplet sizes and the amplitude at which they `stop bouncing' . Isolate droplets by popping unwanted droplets
\bigskip
\begin{center}
\begin{tabu} to 0.8\textwidth { | X[c] | X[c] | }
 \hline
 \textbf{Droplet size(mm)} & \textbf{Sound at which is popped dB}  \\
 \hline
 1  & 70    \\
\hline
\end{tabu}
\end{center}
\bigskip
\end{itemize}

\begin{itemize}
\item Attempt to record walking
\end{itemize}
\bigskip



\textbf{Key findings/observations}

\begin{itemize}
\item Difficult to create and maintain droplets today

\begin{itemize}
\item Temperature, depth of oil may be factors to consider
\end{itemize}
\item Droplets would spontaneously pop - difficult to quantify measurements, better to attempt to observe \& record phenomena instead
\end{itemize}
\bigskip


\textbf{What could be improved?}

\begin{itemize}
\item LED lights along the edge of the petri dish
\item Warm up the oil to reduce viscosity
\item Spinning plate to create artificial relative motion of droplet to the oil surface
\item Microphone above the speaker at fixed distance to measure sound level
\end{itemize}

\bigskip


\textbf{Recommendations, improvements, key pointers, warnings}

\begin{itemize}
\item Double slit to be produced at IoM; clamped and partially submerged in oil bath
\item Larger circular wooden plate to be laser cut
\end{itemize}
