
\section{Experimental Log 16-2}

\textbf{Participants:} SV, KF

\textbf{Location:} Year 1 Teaching Lab, Department of Physics and Astronomy, University College London

\textbf{Start time:} 1000

\textbf{End time:} \ 1200

\bigskip

\textbf{Objectives}

\begin{itemize}
\item To test different liquids
\item To test very thin double slits (made of common plastic - polyethylene)
\item To observe single slit diffraction
\end{itemize}
\bigskip

\textbf{Equipment added/modified; safety precautions}

\begin{itemize}
\item Polyethylene plastic
\item Washing up liquid
\item Vegetable oil
\item Water
\end{itemize}
\bigskip

\textbf{What processes were carried out?}

\begin{itemize}
\item Cut up slits to thin strips and tape to bottom of container
\item Mixed soap water by diluted the washing up liquid with water.
\item Drove at 50 Hz. Observed and recorded standing waves, walking and orbiting phenomena. Changing the amplitude at various points
\item Attempted single slit diffraction by blowing droplets through, eventually changed tactic to moving the slit to the droplet (relative motion)
\end{itemize}
\bigskip

\textbf{Key findings/observations}

\begin{itemize}
\item Temperature: 17.5 degree C
\item Waves are much more visible with washing up liquid (including standing waves!)
\item Broke Faraday instability at 79 dB
\item Observed orbiting
\item Observed walking
\item Droplets form much more easily with soap water
\item Dye affects consistency - however still works. Did not aid visualisation very much
\end{itemize}
\bigskip

\textbf{What could be improved?}

\begin{itemize}
\item Lighting and camera effects
\item Superglue the slits down to observe single \& double slit diffraction
\item Be cautious of the depth (water level)
\item Dye says `shake well'
\end{itemize}

\bigskip


\textbf{Recommendations, improvements, key pointers, warnings}

\begin{itemize}
\item Arrive with high speed camera, lighting and diffuser to record
\item Superglue slits down
\item Perhaps LED slits
\item Lighting model and camera model on Wednesday
\item Start writing up report - similar to formal report style from our experiment - detail out what we've done, make some models and what we've tried
\end{itemize}
