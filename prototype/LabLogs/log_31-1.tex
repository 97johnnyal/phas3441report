
\section{Experimental Log 31-1}

\textbf{Participants:} SV, CKG, MM, BB, KF

\textbf{Location:} Year 1 Teaching Lab, Department of Physics and Astronomy, University College London

\textbf{Start time:} 1000

\textbf{End time:} 1230
\bigskip

\textbf{Objectives}

\begin{itemize}
\item Achieve bouncing droplets with lower viscosity Silicone oil and amplifier
\item Attempt to capture the motion with slow-mo video
\end{itemize}
\bigskip

\textbf{Equipment added/modified; safety precautions}

\begin{itemize}
\item Laser printed box (laser grade plywood - 3mm)

\begin{itemize}
\item Safety: Beware of sharp corners and splinters
\end{itemize}
\item Liquids

\begin{itemize}
\item 50 cSt Silicone oil
\item Washing up liquid
\end{itemize}
\item Hi-Fi amplifier
\item Optical microscope
\item Black paper
\item Tape
\item Wooden Scraps (circular rigid base, spare corner pieces)
\end{itemize}

\bigskip

\textbf{What processes were carried out?}

\begin{itemize}
\item Silicone oil driven at 60 - 80 Hz on mobile app successfully
\item Used wooden circular piece as rigid base, secured with tape
\item Used amplifier with function generator and a self-made wiring system with soldering
\item Function generator set at minimum attenuation and connected via the 4 ohms output
\item Increased volume and bass to achieve high bouncing droplets
\item Recorded slow-mo videos with smartphone
\item Attempt to achieve double slit with improvised diffraction grating
\item But instead the gratings were found to be useful in creating droplets
\item Considered taking images through a microscope
\end{itemize}

\bigskip

\textbf{Key findings/observations}

\begin{itemize}
\item Created bouncing droplets that were sustainable over long duration (see video)
\item Walking droplets (not sure if its actually walking or due to external force)
\item Standing waves when wooden plate allowed to vibrate freely
\item Can combine droplets together, forming crystal lattice
\end{itemize}
\bigskip

\textbf{What could be improved?}

\begin{itemize}
\item Air gaps in between the loudspeaker and wooden base
\item To obtain tools that can clamp the base to the loudspeaker
\item Could also have used an adapter of  2.0 A, 12V
\item Consider using graph paper underneath the droplet to track motion

\begin{itemize}
\item Try to track the amplitude using software from video obtained
\item Quantitative method to record data
\end{itemize}
\item Control droplet size

\begin{itemize}
\item Could produce multiple droplets and pop the ones we do not need
\item Test dye
\end{itemize}
\item Fix petri dish without seeing the tape
\end{itemize}

\bigskip

\textbf{Action plan for next session}

\begin{itemize}
\item Set up in engineering department and record the high speed bouncing droplets
\item Set the correct frame rate to capture only the horizontal motion
\item Get paper clips/graph paper
\item Might need to laser cut another piece of circular base
\end{itemize}
