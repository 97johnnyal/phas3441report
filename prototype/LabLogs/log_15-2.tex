
\section{Experimental Log 15-2}

\textbf{Participants:} SV, KF

\textbf{Location:}: Year 1 Teaching Lab, Department of Physics and Astronomy, University College London

\textbf{Start time:}: 1100

\textbf{End time:} \ 1300


\bigskip

\textbf{Objectives}

\begin{itemize}
\item To observe orbiting pairs, repelling pairs, maximum number of bound droplets
\item To observe double slit diffraction
\item To observe single slit diffraction
\end{itemize}
\bigskip

\textbf{Equipment added/modified; safety precautions}

\begin{itemize}
\item Red food dye
\item Square Tupperware box (135 mm x 135 mm)
\item Square slits of wood to act as diffraction gratings
\item Wooden circular plate (165 mm) for better fitting on loudspeaker
\item Digital thermometer (Digitron T200KC)
\end{itemize}
\bigskip

\textbf{What processes were carried out?}

\begin{itemize}
\item Recorded the temperature of the room
\item Heated up the Silicone oil using the heat sink of a laptop
\item Attached the new circular plate onto mount
\item Taped wooden slits to the Tupperware box.
\item Produced droplets and attempted to drive them through the double slit - slit was too narrow so we expanded the width
\item Reverting back to single slit to test diffraction using round petri dish (also tried with slits submerged)
\item Added dye but it did not diffuse in the Silicone oil.
\end{itemize}
\bigskip

\textbf{Key findings/observations}

\begin{itemize}
\item Previous smaller circular plate was not level
\item Room temperature: 18.9 degrees C (+/- 0.1 degree C)
\item Droplets would die after short time period - perhaps the temperature is the culprit
\item Slits that are non completely submerged have a gradient due to surface tension meaning diffraction via this method is difficult
\item Droplets are not able to pass through the slits due to surface tension, even after tilting the set-up, manually blowing air or pushing with a stick
\end{itemize}
\bigskip

\textbf{What could be improved?}

\begin{itemize}
\item Square Tupperware container isn't flat
\item Difficult to force droplets through slit (used tilting as the compressed air tend to pop the droplets)
\item Reduce the width of slits to reduce surface tension
\end{itemize}

\bigskip


\textbf{Recommendations, improvements, key pointers, warnings}

\begin{itemize}
\item Try different liquids (soap \& vegetable oil). Dye should work in soap water
\item Thin plastic sheet to act as slits
\end{itemize}
