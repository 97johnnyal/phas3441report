\section{Introduction (MM)}
An age-old and much investigated field of research in science has been the nature of light. For centuries, scientists have endeavoured to understand its nature and fundamental properties. At the turn of the 20th century, light was widely believed to behave as a wave, but the discovery of phenomena such as the photoelectric effect called this theory into question. Eventually, it was agreed that light exhibited both wave-like and particle-like behaviour, a phenomenon known as wave-particle duality. The interpretation of this quantum mechanical behaviour, however, is still open to debate.

The commonly accepted interpretation, known as the Copenhagen Interpretation, was first proposed by Niels Bohr and Werner Heisenberg in 1927. It pushed forward the idea that a physical system would only have definite properties upon measurement; prior to measurement it was only possible to measure probabilities of obtaining a given measurement. An alternative interpretation, the De Broglie-Bohm theory, was pushed forwards in 1952. It is deterministic in nature: it claims that even prior to observation there exists a driving equation or `pilot wave' whose properties determine measurements. In 2005, research suggested \cite{couder} that droplets bouncing on a surface could display quantum mechanical behaviour as proposed by pilot wave theory. Indeed, this macroscopic system was a remarkably good analogy for the pilot wave theory; waves induced by a driving droplet quite closely match a driving wave-equation determining the results of measurements. 
\subsection{Aims}\todo{check to see if matches summary}
To this end, the purpose of this project was to develop a cost-effective table top version of the experiment used by \textit{Couder et al.}, demonstrating the various different behaviours of bouncing droplets. Such motion could include simple oscillations, motion parallel to the liquid surface and double slit diffraction of droplet generated waves. It was intended to be paired with a computer based visualisation that can display effects outside the bounds of this apparatus. The overarching goal was to develop a prototype that can be used for outreach purposes, allowing  students to the grasp the non-intuitive nature of quantum mechanics. 

\subsection{Objectives}
The above aims raised the following core project objectives: 
\begin{enumerate}
   \item Development of table top apparatus: 
   \begin{enumerate}
     \item To create a working prototype capable of displaying basic  motion, such as bouncing and walking droplets.
     \item To obtain high speed cameras capable of recording droplet motion
     \item To further improve the prototype, and demonstrate multiple droplet motion (such as droplets repelling and coalescing) and phenomena such as double slit diffraction
    \end{enumerate}
   \item Development of an in-depth simulation:
   \begin{enumerate}
    \item To identify a suitable programming language for simulation purposes
     \item To implement required code in chosen language in order to display droplet motion.
     \item To extend the capabilities of the program in order to cover aspects of quantum mechanics not displayable by the experimental apparatus. Such effects may include double slit diffraction and tunnelling effects, where the droplet escapes through the barrier of a potential well.
    \end{enumerate}
    \item Creating an iteration of the initial prototype usable in college/undergraduate courses as a demonstration tool.  
   \begin{enumerate}
     \item To create a standalone version of the simulation program.
     \item To create a simple to use version of the experimental. apparatus, for third-party users.
     \item To write a business plan for implementation in future.
     \item To explore the possibility of testing the apparatus with undergraduates and A-level students.
    \end{enumerate}
\end{enumerate}
