\section{Conclusion}



In conclusion, only some of the initially stated aims and objectives were accomplished. A basic, table-top version of the apparatus created by \cite{couder} was developed. This apparatus was capable of displaying bouncing, walking and orbiting motion. However, due to time constraints more complex behaviours such as double slit diffraction, were not demonstrable. A suitable programming language was identified, and initial forays into developing a computer simulation were made. Observations from the computer simulation agreed well with the experiment, which suggested the validity of the mathematical model used. More numerical analysis is needed to further confirm this. A pilot outreach lesson with year 12 students was conducted, with generally positive feedback obtained. 

If this project were to be pursued further, with additional resources, further investigations into displaying quantum phenomena with the apparatus could be made. Primarily, phenomena such as double slit diffraction and quantum tunnelling will be demonstar