\section{Conclusion}

In conclusion, only some of the initially stated aims and objectives were accomplished. A basic, table-top version of the apparatus created by \textit{Couder et al.} was developed. This apparatus was capable of displaying bouncing, walking and orbiting motion. However, due to time constraints more complex behaviours such as double slit diffraction, were not demonstrable. A suitable programming language was identified, and initial forays into developing a computer simulation were made. Observations from the computer simulation agreed well with the experiment, which suggested the validity of the mathematical model used. More numerical analysis is needed to further confirm this. A pilot outreach lesson with year 12 students was conducted, with generally positive feedback obtained.

If this project were to be pursued further, with additional resources, further investigations into displaying quantum phenomena with the apparatus could be made. Primarily, phenomena such as double slit diffraction and quantum tunnelling with droplets will be demonstrated, as these allow for more tangible links to be made with quantum mechanics. Quantitative analysis, to definitively verify the mathematical model, alongside development of a standalone version of the simulation is also an another avenue to pursue. In particular, a standalone simulation usable in classrooms would be particularly advantageous, as it allows for increased interactivity, ensuring more effective student engagement. Although the initial outreach efforts were successful, more feedback is necessary from a wider variety of schools in order to make well-supported improvements. If the draft STFC proposal introduced in the report is successfully approved, additional resources will be available for further outreach efforts, and for more effective purpose-oriented improvements to the apparatus such as branding. 